\section{stm32f4xx\+\_\+pwr.\+c}
\label{stm32f4xx__pwr_8c_source}\index{C\+:/\+Users/\+Md. Istiaq Mahbub/\+Desktop/\+I\+M\+U/\+M\+P\+U6050\+\_\+\+Motion\+Driver/\+S\+T\+M32\+F4xx\+\_\+\+Std\+Periph\+\_\+\+Driver/src/stm32f4xx\+\_\+pwr.\+c@{C\+:/\+Users/\+Md. Istiaq Mahbub/\+Desktop/\+I\+M\+U/\+M\+P\+U6050\+\_\+\+Motion\+Driver/\+S\+T\+M32\+F4xx\+\_\+\+Std\+Periph\+\_\+\+Driver/src/stm32f4xx\+\_\+pwr.\+c}}

\begin{DoxyCode}
00001 \textcolor{comment}{/**}
00002 \textcolor{comment}{  ******************************************************************************}
00003 \textcolor{comment}{  * @file    stm32f4xx\_pwr.c}
00004 \textcolor{comment}{  * @author  MCD Application Team}
00005 \textcolor{comment}{  * @version V1.0.0}
00006 \textcolor{comment}{  * @date    30-September-2011}
00007 \textcolor{comment}{  * @brief   This file provides firmware functions to manage the following }
00008 \textcolor{comment}{  *          functionalities of the Power Controller (PWR) peripheral:           }
00009 \textcolor{comment}{  *           - Backup Domain Access}
00010 \textcolor{comment}{  *           - PVD configuration}
00011 \textcolor{comment}{  *           - WakeUp pin configuration}
00012 \textcolor{comment}{  *           - Main and Backup Regulators configuration}
00013 \textcolor{comment}{  *           - FLASH Power Down configuration}
00014 \textcolor{comment}{  *           - Low Power modes configuration}
00015 \textcolor{comment}{  *           - Flags management}
00016 \textcolor{comment}{  *               }
00017 \textcolor{comment}{  ******************************************************************************}
00018 \textcolor{comment}{  * @attention}
00019 \textcolor{comment}{  *}
00020 \textcolor{comment}{  * THE PRESENT FIRMWARE WHICH IS FOR GUIDANCE ONLY AIMS AT PROVIDING CUSTOMERS}
00021 \textcolor{comment}{  * WITH CODING INFORMATION REGARDING THEIR PRODUCTS IN ORDER FOR THEM TO SAVE}
00022 \textcolor{comment}{  * TIME. AS A RESULT, STMICROELECTRONICS SHALL NOT BE HELD LIABLE FOR ANY}
00023 \textcolor{comment}{  * DIRECT, INDIRECT OR CONSEQUENTIAL DAMAGES WITH RESPECT TO ANY CLAIMS ARISING}
00024 \textcolor{comment}{  * FROM THE CONTENT OF SUCH FIRMWARE AND/OR THE USE MADE BY CUSTOMERS OF THE}
00025 \textcolor{comment}{  * CODING INFORMATION CONTAINED HEREIN IN CONNECTION WITH THEIR PRODUCTS.}
00026 \textcolor{comment}{  *}
00027 \textcolor{comment}{  * <h2><center>&copy; COPYRIGHT 2011 STMicroelectronics</center></h2>}
00028 \textcolor{comment}{  ******************************************************************************}
00029 \textcolor{comment}{  */}
00030 
00031 \textcolor{comment}{/* Includes ------------------------------------------------------------------*/}
00032 \textcolor{preprocessor}{#}\textcolor{preprocessor}{include} "stm32f4xx_pwr.h"
00033 \textcolor{preprocessor}{#}\textcolor{preprocessor}{include} "stm32f4xx_rcc.h"
00034 
00035 \textcolor{comment}{/** @addtogroup STM32F4xx\_StdPeriph\_Driver}
00036 \textcolor{comment}{  * @\{}
00037 \textcolor{comment}{  */}
00038 
00039 \textcolor{comment}{/** @defgroup PWR }
00040 \textcolor{comment}{  * @brief PWR driver modules}
00041 \textcolor{comment}{  * @\{}
00042 \textcolor{comment}{  */}
00043 
00044 \textcolor{comment}{/* Private typedef -----------------------------------------------------------*/}
00045 \textcolor{comment}{/* Private define ------------------------------------------------------------*/}
00046 \textcolor{comment}{/* --------- PWR registers bit address in the alias region ---------- */}
00047 \textcolor{preprocessor}{#}\textcolor{preprocessor}{define} \textcolor{preprocessor}{PWR\_OFFSET}               \textcolor{preprocessor}{(}PWR_BASE \textcolor{preprocessor}{-} PERIPH_BASE\textcolor{preprocessor}{)}
00048 
00049 \textcolor{comment}{/* --- CR Register ---*/}
00050 
00051 \textcolor{comment}{/* Alias word address of DBP bit */}
00052 \textcolor{preprocessor}{#}\textcolor{preprocessor}{define} \textcolor{preprocessor}{CR\_OFFSET}                \textcolor{preprocessor}{(}PWR_OFFSET \textcolor{preprocessor}{+} 0x00\textcolor{preprocessor}{)}
00053 \textcolor{preprocessor}{#}\textcolor{preprocessor}{define} \textcolor{preprocessor}{DBP\_BitNumber}            0x08
00054 \textcolor{preprocessor}{#}\textcolor{preprocessor}{define} \textcolor{preprocessor}{CR\_DBP\_BB}                \textcolor{preprocessor}{(}PERIPH_BB_BASE \textcolor{preprocessor}{+} \textcolor{preprocessor}{(}CR_OFFSET \textcolor{preprocessor}{*} 32\textcolor{preprocessor}{)} \textcolor{preprocessor}{+} \textcolor{preprocessor}{(}
      DBP_BitNumber \textcolor{preprocessor}{*} 4\textcolor{preprocessor}{)}\textcolor{preprocessor}{)}
00055 
00056 \textcolor{comment}{/* Alias word address of PVDE bit */}
00057 \textcolor{preprocessor}{#}\textcolor{preprocessor}{define} \textcolor{preprocessor}{PVDE\_BitNumber}           0x04
00058 \textcolor{preprocessor}{#}\textcolor{preprocessor}{define} \textcolor{preprocessor}{CR\_PVDE\_BB}               \textcolor{preprocessor}{(}PERIPH_BB_BASE \textcolor{preprocessor}{+} \textcolor{preprocessor}{(}CR_OFFSET \textcolor{preprocessor}{*} 32\textcolor{preprocessor}{)} \textcolor{preprocessor}{+} \textcolor{preprocessor}{(}
      PVDE_BitNumber \textcolor{preprocessor}{*} 4\textcolor{preprocessor}{)}\textcolor{preprocessor}{)}
00059 
00060 \textcolor{comment}{/* Alias word address of FPDS bit */}
00061 \textcolor{preprocessor}{#}\textcolor{preprocessor}{define} \textcolor{preprocessor}{FPDS\_BitNumber}           0x09
00062 \textcolor{preprocessor}{#}\textcolor{preprocessor}{define} \textcolor{preprocessor}{CR\_FPDS\_BB}               \textcolor{preprocessor}{(}PERIPH_BB_BASE \textcolor{preprocessor}{+} \textcolor{preprocessor}{(}CR_OFFSET \textcolor{preprocessor}{*} 32\textcolor{preprocessor}{)} \textcolor{preprocessor}{+} \textcolor{preprocessor}{(}
      FPDS_BitNumber \textcolor{preprocessor}{*} 4\textcolor{preprocessor}{)}\textcolor{preprocessor}{)}
00063 
00064 \textcolor{comment}{/* Alias word address of PMODE bit */}
00065 \textcolor{preprocessor}{#}\textcolor{preprocessor}{define} \textcolor{preprocessor}{PMODE\_BitNumber}           0x0E
00066 \textcolor{preprocessor}{#}\textcolor{preprocessor}{define} \textcolor{preprocessor}{CR\_PMODE\_BB}               \textcolor{preprocessor}{(}PERIPH_BB_BASE \textcolor{preprocessor}{+} \textcolor{preprocessor}{(}CR_OFFSET \textcolor{preprocessor}{*} 32\textcolor{preprocessor}{)} \textcolor{preprocessor}{+} \textcolor{preprocessor}{(}
      PMODE_BitNumber \textcolor{preprocessor}{*} 4\textcolor{preprocessor}{)}\textcolor{preprocessor}{)}
00067 
00068 
00069 \textcolor{comment}{/* --- CSR Register ---*/}
00070 
00071 \textcolor{comment}{/* Alias word address of EWUP bit */}
00072 \textcolor{preprocessor}{#}\textcolor{preprocessor}{define} \textcolor{preprocessor}{CSR\_OFFSET}               \textcolor{preprocessor}{(}PWR_OFFSET \textcolor{preprocessor}{+} 0x04\textcolor{preprocessor}{)}
00073 \textcolor{preprocessor}{#}\textcolor{preprocessor}{define} \textcolor{preprocessor}{EWUP\_BitNumber}           0x08
00074 \textcolor{preprocessor}{#}\textcolor{preprocessor}{define} \textcolor{preprocessor}{CSR\_EWUP\_BB}              \textcolor{preprocessor}{(}PERIPH_BB_BASE \textcolor{preprocessor}{+} \textcolor{preprocessor}{(}CSR_OFFSET \textcolor{preprocessor}{*} 32\textcolor{preprocessor}{)} \textcolor{preprocessor}{+} \textcolor{preprocessor}{(}
      EWUP_BitNumber \textcolor{preprocessor}{*} 4\textcolor{preprocessor}{)}\textcolor{preprocessor}{)}
00075 
00076 \textcolor{comment}{/* Alias word address of BRE bit */}
00077 \textcolor{preprocessor}{#}\textcolor{preprocessor}{define} \textcolor{preprocessor}{BRE\_BitNumber}            0x09
00078 \textcolor{preprocessor}{#}\textcolor{preprocessor}{define} \textcolor{preprocessor}{CSR\_BRE\_BB}              \textcolor{preprocessor}{(}PERIPH_BB_BASE \textcolor{preprocessor}{+} \textcolor{preprocessor}{(}CSR_OFFSET \textcolor{preprocessor}{*} 32\textcolor{preprocessor}{)} \textcolor{preprocessor}{+} \textcolor{preprocessor}{(}
      BRE_BitNumber \textcolor{preprocessor}{*} 4\textcolor{preprocessor}{)}\textcolor{preprocessor}{)}
00079 
00080 \textcolor{comment}{/* ------------------ PWR registers bit mask ------------------------ */}
00081 
00082 \textcolor{comment}{/* CR register bit mask */}
00083 \textcolor{preprocessor}{#}\textcolor{preprocessor}{define} \textcolor{preprocessor}{CR\_DS\_MASK}               \textcolor{preprocessor}{(}\textcolor{preprocessor}{(}\textcolor{preprocessor}{uint32\_t}\textcolor{preprocessor}{)}0xFFFFFFFC\textcolor{preprocessor}{)}
00084 \textcolor{preprocessor}{#}\textcolor{preprocessor}{define} \textcolor{preprocessor}{CR\_PLS\_MASK}              \textcolor{preprocessor}{(}\textcolor{preprocessor}{(}\textcolor{preprocessor}{uint32\_t}\textcolor{preprocessor}{)}0xFFFFFF1F\textcolor{preprocessor}{)}
00085 
00086 \textcolor{comment}{/* Private macro -------------------------------------------------------------*/}
00087 \textcolor{comment}{/* Private variables ---------------------------------------------------------*/}
00088 \textcolor{comment}{/* Private function prototypes -----------------------------------------------*/}
00089 \textcolor{comment}{/* Private functions ---------------------------------------------------------*/}
00090 
00091 \textcolor{comment}{/** @defgroup PWR\_Private\_Functions}
00092 \textcolor{comment}{  * @\{}
00093 \textcolor{comment}{  */}
00094 
00095 \textcolor{comment}{/** @defgroup PWR\_Group1 Backup Domain Access function }
00096 \textcolor{comment}{ *  @brief   Backup Domain Access function  }
00097 \textcolor{comment}{ *}
00098 \textcolor{comment}{@verbatim   }
00099 \textcolor{comment}{ ===============================================================================}
00100 \textcolor{comment}{                            Backup Domain Access function }
00101 \textcolor{comment}{ ===============================================================================  }
00102 \textcolor{comment}{}
00103 \textcolor{comment}{  After reset, the backup domain (RTC registers, RTC backup data }
00104 \textcolor{comment}{  registers and backup SRAM) is protected against possible unwanted }
00105 \textcolor{comment}{  write accesses. }
00106 \textcolor{comment}{  To enable access to the RTC Domain and RTC registers, proceed as follows:}
00107 \textcolor{comment}{    - Enable the Power Controller (PWR) APB1 interface clock using the}
00108 \textcolor{comment}{      RCC\_APB1PeriphClockCmd() function.}
00109 \textcolor{comment}{    - Enable access to RTC domain using the PWR\_BackupAccessCmd() function.}
00110 \textcolor{comment}{}
00111 \textcolor{comment}{@endverbatim}
00112 \textcolor{comment}{  * @\{}
00113 \textcolor{comment}{  */}
00114 
00115 \textcolor{comment}{/**}
00116 \textcolor{comment}{  * @brief  Deinitializes the PWR peripheral registers to their default reset values.     }
00117 \textcolor{comment}{  * @param  None}
00118 \textcolor{comment}{  * @retval None}
00119 \textcolor{comment}{  */}
00120 \textcolor{keywordtype}{void} PWR_DeInit(\textcolor{keywordtype}{void})
00121 \{
00122   RCC_APB1PeriphResetCmd(RCC_APB1Periph_PWR, ENABLE);
00123   RCC_APB1PeriphResetCmd(RCC_APB1Periph_PWR, DISABLE);
00124 \}
00125 
00126 \textcolor{comment}{/**}
00127 \textcolor{comment}{  * @brief  Enables or disables access to the backup domain (RTC registers, RTC }
00128 \textcolor{comment}{  *         backup data registers and backup SRAM).}
00129 \textcolor{comment}{  * @note   If the HSE divided by 2, 3, ..31 is used as the RTC clock, the }
00130 \textcolor{comment}{  *         Backup Domain Access should be kept enabled.}
00131 \textcolor{comment}{  * @param  NewState: new state of the access to the backup domain.}
00132 \textcolor{comment}{  *          This parameter can be: ENABLE or DISABLE.}
00133 \textcolor{comment}{  * @retval None}
00134 \textcolor{comment}{  */}
00135 \textcolor{keywordtype}{void} PWR_BackupAccessCmd(FunctionalState NewState)
00136 \{
00137   \textcolor{comment}{/* Check the parameters */}
00138   assert_param(IS\_FUNCTIONAL\_STATE(NewState));
00139 
00140   *(\_\_IO uint32\_t *) CR_DBP_BB = (uint32\_t)NewState;
00141 \}
00142 
00143 \textcolor{comment}{/**}
00144 \textcolor{comment}{  * @\}}
00145 \textcolor{comment}{  */}
00146 
00147 \textcolor{comment}{/** @defgroup PWR\_Group2 PVD configuration functions}
00148 \textcolor{comment}{ *  @brief   PVD configuration functions }
00149 \textcolor{comment}{ *}
00150 \textcolor{comment}{@verbatim   }
00151 \textcolor{comment}{ ===============================================================================}
00152 \textcolor{comment}{                           PVD configuration functions}
00153 \textcolor{comment}{ ===============================================================================  }
00154 \textcolor{comment}{}
00155 \textcolor{comment}{ - The PVD is used to monitor the VDD power supply by comparing it to a threshold}
00156 \textcolor{comment}{   selected by the PVD Level (PLS[2:0] bits in the PWR\_CR).}
00157 \textcolor{comment}{ - A PVDO flag is available to indicate if VDD/VDDA is higher or lower than the }
00158 \textcolor{comment}{   PVD threshold. This event is internally connected to the EXTI line16}
00159 \textcolor{comment}{   and can generate an interrupt if enabled through the EXTI registers.}
00160 \textcolor{comment}{ - The PVD is stopped in Standby mode.}
00161 \textcolor{comment}{}
00162 \textcolor{comment}{@endverbatim}
00163 \textcolor{comment}{  * @\{}
00164 \textcolor{comment}{  */}
00165 
00166 \textcolor{comment}{/**}
00167 \textcolor{comment}{  * @brief  Configures the voltage threshold detected by the Power Voltage Detector(PVD).}
00168 \textcolor{comment}{  * @param  PWR\_PVDLevel: specifies the PVD detection level}
00169 \textcolor{comment}{  *          This parameter can be one of the following values:}
00170 \textcolor{comment}{  *            @arg PWR\_PVDLevel\_0: PVD detection level set to 2.0V}
00171 \textcolor{comment}{  *            @arg PWR\_PVDLevel\_1: PVD detection level set to 2.2V}
00172 \textcolor{comment}{  *            @arg PWR\_PVDLevel\_2: PVD detection level set to 2.3V}
00173 \textcolor{comment}{  *            @arg PWR\_PVDLevel\_3: PVD detection level set to 2.5V}
00174 \textcolor{comment}{  *            @arg PWR\_PVDLevel\_4: PVD detection level set to 2.7V}
00175 \textcolor{comment}{  *            @arg PWR\_PVDLevel\_5: PVD detection level set to 2.8V}
00176 \textcolor{comment}{  *            @arg PWR\_PVDLevel\_6: PVD detection level set to 2.9V}
00177 \textcolor{comment}{  *            @arg PWR\_PVDLevel\_7: PVD detection level set to 3.0V}
00178 \textcolor{comment}{  * @note   Refer to the electrical characteristics of you device datasheet for more details. }
00179 \textcolor{comment}{  * @retval None}
00180 \textcolor{comment}{  */}
00181 \textcolor{keywordtype}{void} PWR_PVDLevelConfig(uint32\_t PWR\_PVDLevel)
00182 \{
00183   uint32\_t tmpreg = 0;
00184 
00185   \textcolor{comment}{/* Check the parameters */}
00186   assert_param(IS\_PWR\_PVD\_LEVEL(PWR\_PVDLevel));
00187 
00188   tmpreg = PWR->CR;
00189 
00190   \textcolor{comment}{/* Clear PLS[7:5] bits */}
00191   tmpreg &= CR_PLS_MASK;
00192 
00193   \textcolor{comment}{/* Set PLS[7:5] bits according to PWR\_PVDLevel value */}
00194   tmpreg |= PWR\_PVDLevel;
00195 
00196   \textcolor{comment}{/* Store the new value */}
00197   PWR->CR = tmpreg;
00198 \}
00199 
00200 \textcolor{comment}{/**}
00201 \textcolor{comment}{  * @brief  Enables or disables the Power Voltage Detector(PVD).}
00202 \textcolor{comment}{  * @param  NewState: new state of the PVD.}
00203 \textcolor{comment}{  *         This parameter can be: ENABLE or DISABLE.}
00204 \textcolor{comment}{  * @retval None}
00205 \textcolor{comment}{  */}
00206 \textcolor{keywordtype}{void} PWR_PVDCmd(FunctionalState NewState)
00207 \{
00208   \textcolor{comment}{/* Check the parameters */}
00209   assert_param(IS\_FUNCTIONAL\_STATE(NewState));
00210 
00211   *(\_\_IO uint32\_t *) CR_PVDE_BB = (uint32\_t)NewState;
00212 \}
00213 
00214 \textcolor{comment}{/**}
00215 \textcolor{comment}{  * @\}}
00216 \textcolor{comment}{  */}
00217 
00218 \textcolor{comment}{/** @defgroup PWR\_Group3 WakeUp pin configuration functions}
00219 \textcolor{comment}{ *  @brief   WakeUp pin configuration functions }
00220 \textcolor{comment}{ *}
00221 \textcolor{comment}{@verbatim   }
00222 \textcolor{comment}{ ===============================================================================}
00223 \textcolor{comment}{                    WakeUp pin configuration functions}
00224 \textcolor{comment}{ ===============================================================================  }
00225 \textcolor{comment}{}
00226 \textcolor{comment}{ - WakeUp pin is used to wakeup the system from Standby mode. This pin is }
00227 \textcolor{comment}{   forced in input pull down configuration and is active on rising edges.}
00228 \textcolor{comment}{ - There is only one WakeUp pin: WakeUp Pin 1 on PA.00.}
00229 \textcolor{comment}{}
00230 \textcolor{comment}{@endverbatim}
00231 \textcolor{comment}{  * @\{}
00232 \textcolor{comment}{  */}
00233 
00234 \textcolor{comment}{/**}
00235 \textcolor{comment}{  * @brief  Enables or disables the WakeUp Pin functionality.}
00236 \textcolor{comment}{  * @param  NewState: new state of the WakeUp Pin functionality.}
00237 \textcolor{comment}{  *         This parameter can be: ENABLE or DISABLE.}
00238 \textcolor{comment}{  * @retval None}
00239 \textcolor{comment}{  */}
00240 \textcolor{keywordtype}{void} PWR_WakeUpPinCmd(FunctionalState NewState)
00241 \{
00242   \textcolor{comment}{/* Check the parameters */}
00243   assert_param(IS\_FUNCTIONAL\_STATE(NewState));
00244 
00245   *(\_\_IO uint32\_t *) CSR_EWUP_BB = (uint32\_t)NewState;
00246 \}
00247 
00248 \textcolor{comment}{/**}
00249 \textcolor{comment}{  * @\}}
00250 \textcolor{comment}{  */}
00251 
00252 \textcolor{comment}{/** @defgroup PWR\_Group4 Main and Backup Regulators configuration functions}
00253 \textcolor{comment}{ *  @brief   Main and Backup Regulators configuration functions }
00254 \textcolor{comment}{ *}
00255 \textcolor{comment}{@verbatim   }
00256 \textcolor{comment}{ ===============================================================================}
00257 \textcolor{comment}{                    Main and Backup Regulators configuration functions}
00258 \textcolor{comment}{ ===============================================================================  }
00259 \textcolor{comment}{}
00260 \textcolor{comment}{ - The backup domain includes 4 Kbytes of backup SRAM accessible only from the }
00261 \textcolor{comment}{   CPU, and address in 32-bit, 16-bit or 8-bit mode. Its content is retained }
00262 \textcolor{comment}{   even in Standby or VBAT mode when the low power backup regulator is enabled. }
00263 \textcolor{comment}{   It can be considered as an internal EEPROM when VBAT is always present.}
00264 \textcolor{comment}{   You can use the PWR\_BackupRegulatorCmd() function to enable the low power}
00265 \textcolor{comment}{   backup regulator and use the PWR\_GetFlagStatus(PWR\_FLAG\_BRR) to check if it is}
00266 \textcolor{comment}{   ready or not. }
00267 \textcolor{comment}{}
00268 \textcolor{comment}{ - When the backup domain is supplied by VDD (analog switch connected to VDD) }
00269 \textcolor{comment}{   the backup SRAM is powered from VDD which replaces the VBAT power supply to }
00270 \textcolor{comment}{   save battery life.}
00271 \textcolor{comment}{}
00272 \textcolor{comment}{ - The backup SRAM is not mass erased by an tamper event. It is read protected }
00273 \textcolor{comment}{   to prevent confidential data, such as cryptographic private key, from being }
00274 \textcolor{comment}{   accessed. The backup SRAM can be erased only through the Flash interface when}
00275 \textcolor{comment}{   a protection level change from level 1 to level 0 is requested. }
00276 \textcolor{comment}{   Refer to the description of Read protection (RDP) in the Flash programming manual.}
00277 \textcolor{comment}{}
00278 \textcolor{comment}{ - The main internal regulator can be configured to have a tradeoff between performance}
00279 \textcolor{comment}{   and power consumption when the device does not operate at the maximum frequency. }
00280 \textcolor{comment}{   This is done through PWR\_MainRegulatorModeConfig() function which configure VOS bit}
00281 \textcolor{comment}{   in PWR\_CR register: }
00282 \textcolor{comment}{      - When this bit is set (Regulator voltage output Scale 1 mode selected) the System}
00283 \textcolor{comment}{        frequency can go up to 168 MHz. }
00284 \textcolor{comment}{      - When this bit is reset (Regulator voltage output Scale 2 mode selected) the System}
00285 \textcolor{comment}{        frequency can go up to 144 MHz. }
00286 \textcolor{comment}{   Refer to the datasheets for more details.}
00287 \textcolor{comment}{           }
00288 \textcolor{comment}{@endverbatim}
00289 \textcolor{comment}{  * @\{}
00290 \textcolor{comment}{  */}
00291 
00292 \textcolor{comment}{/**}
00293 \textcolor{comment}{  * @brief  Enables or disables the Backup Regulator.}
00294 \textcolor{comment}{  * @param  NewState: new state of the Backup Regulator.}
00295 \textcolor{comment}{  *          This parameter can be: ENABLE or DISABLE.}
00296 \textcolor{comment}{  * @retval None}
00297 \textcolor{comment}{  */}
00298 \textcolor{keywordtype}{void} PWR_BackupRegulatorCmd(FunctionalState NewState)
00299 \{
00300   \textcolor{comment}{/* Check the parameters */}
00301   assert_param(IS\_FUNCTIONAL\_STATE(NewState));
00302 
00303   *(\_\_IO uint32\_t *) CSR_BRE_BB = (uint32\_t)NewState;
00304 \}
00305 
00306 \textcolor{comment}{/**}
00307 \textcolor{comment}{  * @brief  Configures the main internal regulator output voltage.}
00308 \textcolor{comment}{  * @param  PWR\_Regulator\_Voltage: specifies the regulator output voltage to achieve}
00309 \textcolor{comment}{  *         a tradeoff between performance and power consumption when the device does}
00310 \textcolor{comment}{  *         not operate at the maximum frequency (refer to the datasheets for more details).}
00311 \textcolor{comment}{  *          This parameter can be one of the following values:}
00312 \textcolor{comment}{  *            @arg PWR\_Regulator\_Voltage\_Scale1: Regulator voltage output Scale 1 mode, }
00313 \textcolor{comment}{  *                                                System frequency up to 168 MHz. }
00314 \textcolor{comment}{  *            @arg PWR\_Regulator\_Voltage\_Scale2: Regulator voltage output Scale 2 mode, }
00315 \textcolor{comment}{  *                                                System frequency up to 144 MHz.    }
00316 \textcolor{comment}{  * @retval None}
00317 \textcolor{comment}{  */}
00318 \textcolor{keywordtype}{void} PWR_MainRegulatorModeConfig(uint32\_t PWR\_Regulator\_Voltage)
00319 \{
00320   \textcolor{comment}{/* Check the parameters */}
00321   assert_param(IS\_PWR\_REGULATOR\_VOLTAGE(PWR\_Regulator\_Voltage));
00322 
00323   \textcolor{keywordflow}{if} (PWR\_Regulator\_Voltage == PWR_Regulator_Voltage_Scale2)
00324   \{
00325     PWR->CR &= ~PWR_Regulator_Voltage_Scale1;
00326   \}
00327   \textcolor{keywordflow}{else}
00328   \{
00329     PWR->CR |= PWR_Regulator_Voltage_Scale1;
00330   \}
00331 \}
00332 
00333 \textcolor{comment}{/**}
00334 \textcolor{comment}{  * @\}}
00335 \textcolor{comment}{  */}
00336 
00337 \textcolor{comment}{/** @defgroup PWR\_Group5 FLASH Power Down configuration functions}
00338 \textcolor{comment}{ *  @brief   FLASH Power Down configuration functions }
00339 \textcolor{comment}{ *}
00340 \textcolor{comment}{@verbatim   }
00341 \textcolor{comment}{ ===============================================================================}
00342 \textcolor{comment}{           FLASH Power Down configuration functions}
00343 \textcolor{comment}{ ===============================================================================  }
00344 \textcolor{comment}{}
00345 \textcolor{comment}{ - By setting the FPDS bit in the PWR\_CR register by using the PWR\_FlashPowerDownCmd()}
00346 \textcolor{comment}{   function, the Flash memory also enters power down mode when the device enters }
00347 \textcolor{comment}{   Stop mode. When the Flash memory is in power down mode, an additional startup }
00348 \textcolor{comment}{   delay is incurred when waking up from Stop mode.}
00349 \textcolor{comment}{}
00350 \textcolor{comment}{@endverbatim}
00351 \textcolor{comment}{  * @\{}
00352 \textcolor{comment}{  */}
00353 
00354 \textcolor{comment}{/**}
00355 \textcolor{comment}{  * @brief  Enables or disables the Flash Power Down in STOP mode.}
00356 \textcolor{comment}{  * @param  NewState: new state of the Flash power mode.}
00357 \textcolor{comment}{  *          This parameter can be: ENABLE or DISABLE.}
00358 \textcolor{comment}{  * @retval None}
00359 \textcolor{comment}{  */}
00360 \textcolor{keywordtype}{void} PWR_FlashPowerDownCmd(FunctionalState NewState)
00361 \{
00362   \textcolor{comment}{/* Check the parameters */}
00363   assert_param(IS\_FUNCTIONAL\_STATE(NewState));
00364 
00365   *(\_\_IO uint32\_t *) CR_FPDS_BB = (uint32\_t)NewState;
00366 \}
00367 
00368 \textcolor{comment}{/**}
00369 \textcolor{comment}{  * @\}}
00370 \textcolor{comment}{  */}
00371 
00372 \textcolor{comment}{/** @defgroup PWR\_Group6 Low Power modes configuration functions}
00373 \textcolor{comment}{ *  @brief   Low Power modes configuration functions }
00374 \textcolor{comment}{ *}
00375 \textcolor{comment}{@verbatim   }
00376 \textcolor{comment}{ ===============================================================================}
00377 \textcolor{comment}{                    Low Power modes configuration functions}
00378 \textcolor{comment}{ ===============================================================================  }
00379 \textcolor{comment}{}
00380 \textcolor{comment}{  The devices feature 3 low-power modes:}
00381 \textcolor{comment}{   - Sleep mode: Cortex-M4 core stopped, peripherals kept running.}
00382 \textcolor{comment}{   - Stop mode: all clocks are stopped, regulator running, regulator in low power mode}
00383 \textcolor{comment}{   - Standby mode: 1.2V domain powered off.}
00384 \textcolor{comment}{   }
00385 \textcolor{comment}{   Sleep mode}
00386 \textcolor{comment}{   ===========}
00387 \textcolor{comment}{    - Entry:}
00388 \textcolor{comment}{      - The Sleep mode is entered by using the \_\_WFI() or \_\_WFE() functions.}
00389 \textcolor{comment}{    - Exit:}
00390 \textcolor{comment}{      - Any peripheral interrupt acknowledged by the nested vectored interrupt }
00391 \textcolor{comment}{        controller (NVIC) can wake up the device from Sleep mode.}
00392 \textcolor{comment}{}
00393 \textcolor{comment}{   Stop mode}
00394 \textcolor{comment}{   ==========}
00395 \textcolor{comment}{   In Stop mode, all clocks in the 1.2V domain are stopped, the PLL, the HSI,}
00396 \textcolor{comment}{   and the HSE RC oscillators are disabled. Internal SRAM and register contents }
00397 \textcolor{comment}{   are preserved.}
00398 \textcolor{comment}{   The voltage regulator can be configured either in normal or low-power mode.}
00399 \textcolor{comment}{   To minimize the consumption In Stop mode, FLASH can be powered off before }
00400 \textcolor{comment}{   entering the Stop mode. It can be switched on again by software after exiting }
00401 \textcolor{comment}{   the Stop mode using the PWR\_FlashPowerDownCmd() function. }
00402 \textcolor{comment}{   }
00403 \textcolor{comment}{    - Entry:}
00404 \textcolor{comment}{      - The Stop mode is entered using the PWR\_EnterSTOPMode(PWR\_Regulator\_LowPower,) }
00405 \textcolor{comment}{        function with regulator in LowPower or with Regulator ON.}
00406 \textcolor{comment}{    - Exit:}
00407 \textcolor{comment}{      - Any EXTI Line (Internal or External) configured in Interrupt/Event mode.}
00408 \textcolor{comment}{      }
00409 \textcolor{comment}{   Standby mode}
00410 \textcolor{comment}{   ============}
00411 \textcolor{comment}{   The Standby mode allows to achieve the lowest power consumption. It is based }
00412 \textcolor{comment}{   on the Cortex-M4 deepsleep mode, with the voltage regulator disabled. }
00413 \textcolor{comment}{   The 1.2V domain is consequently powered off. The PLL, the HSI oscillator and }
00414 \textcolor{comment}{   the HSE oscillator are also switched off. SRAM and register contents are lost }
00415 \textcolor{comment}{   except for the RTC registers, RTC backup registers, backup SRAM and Standby }
00416 \textcolor{comment}{   circuitry.}
00417 \textcolor{comment}{   }
00418 \textcolor{comment}{   The voltage regulator is OFF.}
00419 \textcolor{comment}{      }
00420 \textcolor{comment}{    - Entry:}
00421 \textcolor{comment}{      - The Standby mode is entered using the PWR\_EnterSTANDBYMode() function.}
00422 \textcolor{comment}{    - Exit:}
00423 \textcolor{comment}{      - WKUP pin rising edge, RTC alarm (Alarm A and Alarm B), RTC wakeup,}
00424 \textcolor{comment}{        tamper event, time-stamp event, external reset in NRST pin, IWDG reset.              }
00425 \textcolor{comment}{}
00426 \textcolor{comment}{   Auto-wakeup (AWU) from low-power mode}
00427 \textcolor{comment}{   =====================================}
00428 \textcolor{comment}{   The MCU can be woken up from low-power mode by an RTC Alarm event, an RTC }
00429 \textcolor{comment}{   Wakeup event, a tamper event, a time-stamp event, or a comparator event, }
00430 \textcolor{comment}{   without depending on an external interrupt (Auto-wakeup mode).}
00431 \textcolor{comment}{}
00432 \textcolor{comment}{   - RTC auto-wakeup (AWU) from the Stop mode}
00433 \textcolor{comment}{     ----------------------------------------}
00434 \textcolor{comment}{     }
00435 \textcolor{comment}{     - To wake up from the Stop mode with an RTC alarm event, it is necessary to:}
00436 \textcolor{comment}{       - Configure the EXTI Line 17 to be sensitive to rising edges (Interrupt }
00437 \textcolor{comment}{         or Event modes) using the EXTI\_Init() function.}
00438 \textcolor{comment}{       - Enable the RTC Alarm Interrupt using the RTC\_ITConfig() function}
00439 \textcolor{comment}{       - Configure the RTC to generate the RTC alarm using the RTC\_SetAlarm() }
00440 \textcolor{comment}{         and RTC\_AlarmCmd() functions.}
00441 \textcolor{comment}{     - To wake up from the Stop mode with an RTC Tamper or time stamp event, it }
00442 \textcolor{comment}{       is necessary to:}
00443 \textcolor{comment}{       - Configure the EXTI Line 21 to be sensitive to rising edges (Interrupt }
00444 \textcolor{comment}{         or Event modes) using the EXTI\_Init() function.}
00445 \textcolor{comment}{       - Enable the RTC Tamper or time stamp Interrupt using the RTC\_ITConfig() }
00446 \textcolor{comment}{         function}
00447 \textcolor{comment}{       - Configure the RTC to detect the tamper or time stamp event using the}
00448 \textcolor{comment}{         RTC\_TimeStampConfig(), RTC\_TamperTriggerConfig() and RTC\_TamperCmd()}
00449 \textcolor{comment}{         functions.}
00450 \textcolor{comment}{     - To wake up from the Stop mode with an RTC WakeUp event, it is necessary to:}
00451 \textcolor{comment}{       - Configure the EXTI Line 22 to be sensitive to rising edges (Interrupt }
00452 \textcolor{comment}{         or Event modes) using the EXTI\_Init() function.}
00453 \textcolor{comment}{       - Enable the RTC WakeUp Interrupt using the RTC\_ITConfig() function}
00454 \textcolor{comment}{       - Configure the RTC to generate the RTC WakeUp event using the RTC\_WakeUpClockConfig(), }
00455 \textcolor{comment}{         RTC\_SetWakeUpCounter() and RTC\_WakeUpCmd() functions.}
00456 \textcolor{comment}{}
00457 \textcolor{comment}{   - RTC auto-wakeup (AWU) from the Standby mode}
00458 \textcolor{comment}{     -------------------------------------------}
00459 \textcolor{comment}{     - To wake up from the Standby mode with an RTC alarm event, it is necessary to:}
00460 \textcolor{comment}{       - Enable the RTC Alarm Interrupt using the RTC\_ITConfig() function}
00461 \textcolor{comment}{       - Configure the RTC to generate the RTC alarm using the RTC\_SetAlarm() }
00462 \textcolor{comment}{         and RTC\_AlarmCmd() functions.}
00463 \textcolor{comment}{     - To wake up from the Standby mode with an RTC Tamper or time stamp event, it }
00464 \textcolor{comment}{       is necessary to:}
00465 \textcolor{comment}{       - Enable the RTC Tamper or time stamp Interrupt using the RTC\_ITConfig() }
00466 \textcolor{comment}{         function}
00467 \textcolor{comment}{       - Configure the RTC to detect the tamper or time stamp event using the}
00468 \textcolor{comment}{         RTC\_TimeStampConfig(), RTC\_TamperTriggerConfig() and RTC\_TamperCmd()}
00469 \textcolor{comment}{         functions.}
00470 \textcolor{comment}{     - To wake up from the Standby mode with an RTC WakeUp event, it is necessary to:}
00471 \textcolor{comment}{       - Enable the RTC WakeUp Interrupt using the RTC\_ITConfig() function}
00472 \textcolor{comment}{       - Configure the RTC to generate the RTC WakeUp event using the RTC\_WakeUpClockConfig(), }
00473 \textcolor{comment}{         RTC\_SetWakeUpCounter() and RTC\_WakeUpCmd() functions.}
00474 \textcolor{comment}{}
00475 \textcolor{comment}{@endverbatim}
00476 \textcolor{comment}{  * @\{}
00477 \textcolor{comment}{  */}
00478 
00479 \textcolor{comment}{/**}
00480 \textcolor{comment}{  * @brief  Enters STOP mode.}
00481 \textcolor{comment}{  *   }
00482 \textcolor{comment}{  * @note   In Stop mode, all I/O pins keep the same state as in Run mode.}
00483 \textcolor{comment}{  * @note   When exiting Stop mode by issuing an interrupt or a wakeup event, }
00484 \textcolor{comment}{  *         the HSI RC oscillator is selected as system clock.}
00485 \textcolor{comment}{  * @note   When the voltage regulator operates in low power mode, an additional }
00486 \textcolor{comment}{  *         startup delay is incurred when waking up from Stop mode. }
00487 \textcolor{comment}{  *         By keeping the internal regulator ON during Stop mode, the consumption }
00488 \textcolor{comment}{  *         is higher although the startup time is reduced.           }
00489 \textcolor{comment}{  *     }
00490 \textcolor{comment}{  * @param  PWR\_Regulator: specifies the regulator state in STOP mode.}
00491 \textcolor{comment}{  *          This parameter can be one of the following values:}
00492 \textcolor{comment}{  *            @arg PWR\_Regulator\_ON: STOP mode with regulator ON}
00493 \textcolor{comment}{  *            @arg PWR\_Regulator\_LowPower: STOP mode with regulator in low power mode}
00494 \textcolor{comment}{  * @param  PWR\_STOPEntry: specifies if STOP mode in entered with WFI or WFE instruction.}
00495 \textcolor{comment}{  *          This parameter can be one of the following values:}
00496 \textcolor{comment}{  *            @arg PWR\_STOPEntry\_WFI: enter STOP mode with WFI instruction}
00497 \textcolor{comment}{  *            @arg PWR\_STOPEntry\_WFE: enter STOP mode with WFE instruction}
00498 \textcolor{comment}{  * @retval None}
00499 \textcolor{comment}{  */}
00500 \textcolor{keywordtype}{void} PWR_EnterSTOPMode(uint32\_t PWR\_Regulator, uint8\_t PWR\_STOPEntry)
00501 \{
00502   uint32\_t tmpreg = 0;
00503 
00504   \textcolor{comment}{/* Check the parameters */}
00505   assert_param(IS\_PWR\_REGULATOR(PWR\_Regulator));
00506   assert_param(IS\_PWR\_STOP\_ENTRY(PWR\_STOPEntry));
00507 
00508   \textcolor{comment}{/* Select the regulator state in STOP mode ---------------------------------*/}
00509   tmpreg = PWR->CR;
00510   \textcolor{comment}{/* Clear PDDS and LPDSR bits */}
00511   tmpreg &= CR_DS_MASK;
00512 
00513   \textcolor{comment}{/* Set LPDSR bit according to PWR\_Regulator value */}
00514   tmpreg |= PWR\_Regulator;
00515 
00516   \textcolor{comment}{/* Store the new value */}
00517   PWR->CR = tmpreg;
00518 
00519   \textcolor{comment}{/* Set SLEEPDEEP bit of Cortex System Control Register */}
00520   SCB->SCR |= SCB\_SCR\_SLEEPDEEP\_Msk;
00521 
00522   \textcolor{comment}{/* Select STOP mode entry --------------------------------------------------*/}
00523   \textcolor{keywordflow}{if}(PWR\_STOPEntry == PWR_STOPEntry_WFI)
00524   \{
00525     \textcolor{comment}{/* Request Wait For Interrupt */}
00526     \_\_WFI();
00527   \}
00528   \textcolor{keywordflow}{else}
00529   \{
00530     \textcolor{comment}{/* Request Wait For Event */}
00531     \_\_WFE();
00532   \}
00533   \textcolor{comment}{/* Reset SLEEPDEEP bit of Cortex System Control Register */}
00534   SCB->SCR &= (uint32\_t)~((uint32\_t)SCB\_SCR\_SLEEPDEEP\_Msk);
00535 \}
00536 
00537 \textcolor{comment}{/**}
00538 \textcolor{comment}{  * @brief  Enters STANDBY mode.}
00539 \textcolor{comment}{  * @note   In Standby mode, all I/O pins are high impedance except for:}
00540 \textcolor{comment}{  *          - Reset pad (still available) }
00541 \textcolor{comment}{  *          - RTC\_AF1 pin (PC13) if configured for tamper, time-stamp, RTC }
00542 \textcolor{comment}{  *            Alarm out, or RTC clock calibration out.}
00543 \textcolor{comment}{  *          - RTC\_AF2 pin (PI8) if configured for tamper or time-stamp.  }
00544 \textcolor{comment}{  *          - WKUP pin 1 (PA0) if enabled.       }
00545 \textcolor{comment}{  * @param  None}
00546 \textcolor{comment}{  * @retval None}
00547 \textcolor{comment}{  */}
00548 \textcolor{keywordtype}{void} PWR_EnterSTANDBYMode(\textcolor{keywordtype}{void})
00549 \{
00550   \textcolor{comment}{/* Clear Wakeup flag */}
00551   PWR->CR |= PWR_CR_CWUF;
00552 
00553   \textcolor{comment}{/* Select STANDBY mode */}
00554   PWR->CR |= PWR_CR_PDDS;
00555 
00556   \textcolor{comment}{/* Set SLEEPDEEP bit of Cortex System Control Register */}
00557   SCB->SCR |= SCB\_SCR\_SLEEPDEEP\_Msk;
00558 
00559 \textcolor{comment}{/* This option is used to ensure that store operations are completed */}
00560 \textcolor{preprocessor}{#}\textcolor{preprocessor}{if} \textcolor{preprocessor}{defined} \textcolor{preprocessor}{(} \textcolor{preprocessor}{\_\_CC\_ARM}   \textcolor{preprocessor}{)}
00561   \_\_force\_stores();
00562 \textcolor{preprocessor}{#}\textcolor{preprocessor}{endif}
00563   \textcolor{comment}{/* Request Wait For Interrupt */}
00564   \_\_WFI();
00565 \}
00566 
00567 \textcolor{comment}{/**}
00568 \textcolor{comment}{  * @\}}
00569 \textcolor{comment}{  */}
00570 
00571 \textcolor{comment}{/** @defgroup PWR\_Group7 Flags management functions}
00572 \textcolor{comment}{ *  @brief   Flags management functions }
00573 \textcolor{comment}{ *}
00574 \textcolor{comment}{@verbatim   }
00575 \textcolor{comment}{ ===============================================================================}
00576 \textcolor{comment}{                           Flags management functions}
00577 \textcolor{comment}{ ===============================================================================  }
00578 \textcolor{comment}{}
00579 \textcolor{comment}{@endverbatim}
00580 \textcolor{comment}{  * @\{}
00581 \textcolor{comment}{  */}
00582 
00583 \textcolor{comment}{/**}
00584 \textcolor{comment}{  * @brief  Checks whether the specified PWR flag is set or not.}
00585 \textcolor{comment}{  * @param  PWR\_FLAG: specifies the flag to check.}
00586 \textcolor{comment}{  *          This parameter can be one of the following values:}
00587 \textcolor{comment}{  *            @arg PWR\_FLAG\_WU: Wake Up flag. This flag indicates that a wakeup event }
00588 \textcolor{comment}{  *                  was received from the WKUP pin or from the RTC alarm (Alarm A }
00589 \textcolor{comment}{  *                  or Alarm B), RTC Tamper event, RTC TimeStamp event or RTC Wakeup.}
00590 \textcolor{comment}{  *                  An additional wakeup event is detected if the WKUP pin is enabled }
00591 \textcolor{comment}{  *                  (by setting the EWUP bit) when the WKUP pin level is already high.  }
00592 \textcolor{comment}{  *            @arg PWR\_FLAG\_SB: StandBy flag. This flag indicates that the system was}
00593 \textcolor{comment}{  *                  resumed from StandBy mode.    }
00594 \textcolor{comment}{  *            @arg PWR\_FLAG\_PVDO: PVD Output. This flag is valid only if PVD is enabled }
00595 \textcolor{comment}{  *                  by the PWR\_PVDCmd() function. The PVD is stopped by Standby mode }
00596 \textcolor{comment}{  *                  For this reason, this bit is equal to 0 after Standby or reset}
00597 \textcolor{comment}{  *                  until the PVDE bit is set.}
00598 \textcolor{comment}{  *            @arg PWR\_FLAG\_BRR: Backup regulator ready flag. This bit is not reset }
00599 \textcolor{comment}{  *                  when the device wakes up from Standby mode or by a system reset }
00600 \textcolor{comment}{  *                  or power reset.  }
00601 \textcolor{comment}{  *            @arg PWR\_FLAG\_VOSRDY: This flag indicates that the Regulator voltage }
00602 \textcolor{comment}{  *                 scaling output selection is ready. }
00603 \textcolor{comment}{  * @retval The new state of PWR\_FLAG (SET or RESET).}
00604 \textcolor{comment}{  */}
00605 FlagStatus PWR_GetFlagStatus(uint32\_t PWR\_FLAG)
00606 \{
00607   FlagStatus bitstatus = RESET;
00608 
00609   \textcolor{comment}{/* Check the parameters */}
00610   assert_param(IS\_PWR\_GET\_FLAG(PWR\_FLAG));
00611 
00612   \textcolor{keywordflow}{if} ((PWR->CSR & PWR\_FLAG) != (uint32\_t)RESET)
00613   \{
00614     bitstatus = SET;
00615   \}
00616   \textcolor{keywordflow}{else}
00617   \{
00618     bitstatus = RESET;
00619   \}
00620   \textcolor{comment}{/* Return the flag status */}
00621   \textcolor{keywordflow}{return} bitstatus;
00622 \}
00623 
00624 \textcolor{comment}{/**}
00625 \textcolor{comment}{  * @brief  Clears the PWR's pending flags.}
00626 \textcolor{comment}{  * @param  PWR\_FLAG: specifies the flag to clear.}
00627 \textcolor{comment}{  *          This parameter can be one of the following values:}
00628 \textcolor{comment}{  *            @arg PWR\_FLAG\_WU: Wake Up flag}
00629 \textcolor{comment}{  *            @arg PWR\_FLAG\_SB: StandBy flag}
00630 \textcolor{comment}{  * @retval None}
00631 \textcolor{comment}{  */}
00632 \textcolor{keywordtype}{void} PWR_ClearFlag(uint32\_t PWR\_FLAG)
00633 \{
00634   \textcolor{comment}{/* Check the parameters */}
00635   assert_param(IS\_PWR\_CLEAR\_FLAG(PWR\_FLAG));
00636 
00637   PWR->CR |=  PWR\_FLAG << 2;
00638 \}
00639 
00640 \textcolor{comment}{/**}
00641 \textcolor{comment}{  * @\}}
00642 \textcolor{comment}{  */}
00643 
00644 \textcolor{comment}{/**}
00645 \textcolor{comment}{  * @\}}
00646 \textcolor{comment}{  */}
00647 
00648 \textcolor{comment}{/**}
00649 \textcolor{comment}{  * @\}}
00650 \textcolor{comment}{  */}
00651 
00652 \textcolor{comment}{/**}
00653 \textcolor{comment}{  * @\}}
00654 \textcolor{comment}{  */}
00655 
00656 \textcolor{comment}{/******************* (C) COPYRIGHT 2011 STMicroelectronics *****END OF FILE****/}
\end{DoxyCode}
