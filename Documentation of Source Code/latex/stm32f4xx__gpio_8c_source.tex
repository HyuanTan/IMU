\section{stm32f4xx\+\_\+gpio.\+c}
\label{stm32f4xx__gpio_8c_source}\index{C\+:/\+Users/\+Md. Istiaq Mahbub/\+Desktop/\+I\+M\+U/\+M\+P\+U6050\+\_\+\+Motion\+Driver/\+S\+T\+M32\+F4xx\+\_\+\+Std\+Periph\+\_\+\+Driver/src/stm32f4xx\+\_\+gpio.\+c@{C\+:/\+Users/\+Md. Istiaq Mahbub/\+Desktop/\+I\+M\+U/\+M\+P\+U6050\+\_\+\+Motion\+Driver/\+S\+T\+M32\+F4xx\+\_\+\+Std\+Periph\+\_\+\+Driver/src/stm32f4xx\+\_\+gpio.\+c}}

\begin{DoxyCode}
00001 \textcolor{comment}{/**}
00002 \textcolor{comment}{  ******************************************************************************}
00003 \textcolor{comment}{  * @file    stm32f4xx\_gpio.c}
00004 \textcolor{comment}{  * @author  MCD Application Team}
00005 \textcolor{comment}{  * @version V1.0.0}
00006 \textcolor{comment}{  * @date    30-September-2011}
00007 \textcolor{comment}{  * @brief   This file provides firmware functions to manage the following }
00008 \textcolor{comment}{  *          functionalities of the GPIO peripheral:           }
00009 \textcolor{comment}{  *           - Initialization and Configuration}
00010 \textcolor{comment}{  *           - GPIO Read and Write}
00011 \textcolor{comment}{  *           - GPIO Alternate functions configuration}
00012 \textcolor{comment}{  * }
00013 \textcolor{comment}{  *  @verbatim}
00014 \textcolor{comment}{  *}
00015 \textcolor{comment}{  *          ===================================================================}
00016 \textcolor{comment}{  *                                 How to use this driver}
00017 \textcolor{comment}{  *          ===================================================================       }
00018 \textcolor{comment}{  *           1. Enable the GPIO AHB clock using the following function}
00019 \textcolor{comment}{  *                RCC\_AHB1PeriphClockCmd(RCC\_AHB1Periph\_GPIOx, ENABLE);}
00020 \textcolor{comment}{  *             }
00021 \textcolor{comment}{  *           2. Configure the GPIO pin(s) using GPIO\_Init()}
00022 \textcolor{comment}{  *              Four possible configuration are available for each pin:}
00023 \textcolor{comment}{  *                - Input: Floating, Pull-up, Pull-down.}
00024 \textcolor{comment}{  *                - Output: Push-Pull (Pull-up, Pull-down or no Pull)}
00025 \textcolor{comment}{  *                          Open Drain (Pull-up, Pull-down or no Pull).}
00026 \textcolor{comment}{  *                  In output mode, the speed is configurable: 2 MHz, 25 MHz,}
00027 \textcolor{comment}{  *                  50 MHz or 100 MHz.}
00028 \textcolor{comment}{  *                - Alternate Function: Push-Pull (Pull-up, Pull-down or no Pull)}
00029 \textcolor{comment}{  *                                      Open Drain (Pull-up, Pull-down or no Pull).}
00030 \textcolor{comment}{  *                - Analog: required mode when a pin is to be used as ADC channel}
00031 \textcolor{comment}{  *                          or DAC output.}
00032 \textcolor{comment}{  * }
00033 \textcolor{comment}{  *          3- Peripherals alternate function:}
00034 \textcolor{comment}{  *              - For ADC and DAC, configure the desired pin in analog mode using }
00035 \textcolor{comment}{  *                  GPIO\_InitStruct->GPIO\_Mode = GPIO\_Mode\_AN;}
00036 \textcolor{comment}{  *              - For other peripherals (TIM, USART...):}
00037 \textcolor{comment}{  *                 - Connect the pin to the desired peripherals' Alternate }
00038 \textcolor{comment}{  *                   Function (AF) using GPIO\_PinAFConfig() function}
00039 \textcolor{comment}{  *                 - Configure the desired pin in alternate function mode using}
00040 \textcolor{comment}{  *                   GPIO\_InitStruct->GPIO\_Mode = GPIO\_Mode\_AF}
00041 \textcolor{comment}{  *                 - Select the type, pull-up/pull-down and output speed via }
00042 \textcolor{comment}{  *                   GPIO\_PuPd, GPIO\_OType and GPIO\_Speed members}
00043 \textcolor{comment}{  *                 - Call GPIO\_Init() function}
00044 \textcolor{comment}{  *        }
00045 \textcolor{comment}{  *          4. To get the level of a pin configured in input mode use GPIO\_ReadInputDataBit()}
00046 \textcolor{comment}{  *          }
00047 \textcolor{comment}{  *          5. To set/reset the level of a pin configured in output mode use}
00048 \textcolor{comment}{  *             GPIO\_SetBits()/GPIO\_ResetBits()}
00049 \textcolor{comment}{  *               }
00050 \textcolor{comment}{  *          6. During and just after reset, the alternate functions are not }
00051 \textcolor{comment}{  *             active and the GPIO pins are configured in input floating mode}
00052 \textcolor{comment}{  *             (except JTAG pins).}
00053 \textcolor{comment}{  *}
00054 \textcolor{comment}{  *          7. The LSE oscillator pins OSC32\_IN and OSC32\_OUT can be used as }
00055 \textcolor{comment}{  *             general-purpose (PC14 and PC15, respectively) when the LSE}
00056 \textcolor{comment}{  *             oscillator is off. The LSE has priority over the GPIO function.}
00057 \textcolor{comment}{  *}
00058 \textcolor{comment}{  *          8. The HSE oscillator pins OSC\_IN/OSC\_OUT can be used as }
00059 \textcolor{comment}{  *             general-purpose PH0 and PH1, respectively, when the HSE }
00060 \textcolor{comment}{  *             oscillator is off. The HSE has priority over the GPIO function.}
00061 \textcolor{comment}{  *             }
00062 \textcolor{comment}{  *  @endverbatim        }
00063 \textcolor{comment}{  *}
00064 \textcolor{comment}{  ******************************************************************************}
00065 \textcolor{comment}{  * @attention}
00066 \textcolor{comment}{  *}
00067 \textcolor{comment}{  * THE PRESENT FIRMWARE WHICH IS FOR GUIDANCE ONLY AIMS AT PROVIDING CUSTOMERS}
00068 \textcolor{comment}{  * WITH CODING INFORMATION REGARDING THEIR PRODUCTS IN ORDER FOR THEM TO SAVE}
00069 \textcolor{comment}{  * TIME. AS A RESULT, STMICROELECTRONICS SHALL NOT BE HELD LIABLE FOR ANY}
00070 \textcolor{comment}{  * DIRECT, INDIRECT OR CONSEQUENTIAL DAMAGES WITH RESPECT TO ANY CLAIMS ARISING}
00071 \textcolor{comment}{  * FROM THE CONTENT OF SUCH FIRMWARE AND/OR THE USE MADE BY CUSTOMERS OF THE}
00072 \textcolor{comment}{  * CODING INFORMATION CONTAINED HEREIN IN CONNECTION WITH THEIR PRODUCTS.}
00073 \textcolor{comment}{  *}
00074 \textcolor{comment}{  * <h2><center>&copy; COPYRIGHT 2011 STMicroelectronics</center></h2>}
00075 \textcolor{comment}{  ******************************************************************************}
00076 \textcolor{comment}{  */}
00077 
00078 \textcolor{comment}{/* Includes ------------------------------------------------------------------*/}
00079 \textcolor{preprocessor}{#}\textcolor{preprocessor}{include} "stm32f4xx_gpio.h"
00080 \textcolor{preprocessor}{#}\textcolor{preprocessor}{include} "stm32f4xx_rcc.h"
00081 
00082 \textcolor{comment}{/** @addtogroup STM32F4xx\_StdPeriph\_Driver}
00083 \textcolor{comment}{  * @\{}
00084 \textcolor{comment}{  */}
00085 
00086 \textcolor{comment}{/** @defgroup GPIO }
00087 \textcolor{comment}{  * @brief GPIO driver modules}
00088 \textcolor{comment}{  * @\{}
00089 \textcolor{comment}{  */}
00090 
00091 \textcolor{comment}{/* Private typedef -----------------------------------------------------------*/}
00092 \textcolor{comment}{/* Private define ------------------------------------------------------------*/}
00093 \textcolor{comment}{/* Private macro -------------------------------------------------------------*/}
00094 \textcolor{comment}{/* Private variables ---------------------------------------------------------*/}
00095 \textcolor{comment}{/* Private function prototypes -----------------------------------------------*/}
00096 \textcolor{comment}{/* Private functions ---------------------------------------------------------*/}
00097 
00098 \textcolor{comment}{/** @defgroup GPIO\_Private\_Functions}
00099 \textcolor{comment}{  * @\{}
00100 \textcolor{comment}{  */}
00101 
00102 \textcolor{comment}{/** @defgroup GPIO\_Group1 Initialization and Configuration}
00103 \textcolor{comment}{ *  @brief   Initialization and Configuration}
00104 \textcolor{comment}{ *}
00105 \textcolor{comment}{@verbatim   }
00106 \textcolor{comment}{ ===============================================================================}
00107 \textcolor{comment}{                        Initialization and Configuration}
00108 \textcolor{comment}{ ===============================================================================  }
00109 \textcolor{comment}{}
00110 \textcolor{comment}{@endverbatim}
00111 \textcolor{comment}{  * @\{}
00112 \textcolor{comment}{  */}
00113 
00114 \textcolor{comment}{/**}
00115 \textcolor{comment}{  * @brief  Deinitializes the GPIOx peripheral registers to their default reset values.}
00116 \textcolor{comment}{  * @note   By default, The GPIO pins are configured in input floating mode (except JTAG pins).}
00117 \textcolor{comment}{  * @param  GPIOx: where x can be (A..I) to select the GPIO peripheral.}
00118 \textcolor{comment}{  * @retval None}
00119 \textcolor{comment}{  */}
00120 \textcolor{keywordtype}{void} GPIO_DeInit(GPIO\_TypeDef* GPIOx)
00121 \{
00122   \textcolor{comment}{/* Check the parameters */}
00123   assert_param(IS\_GPIO\_ALL\_PERIPH(GPIOx));
00124 
00125   \textcolor{keywordflow}{if} (GPIOx == GPIOA)
00126   \{
00127     RCC_AHB1PeriphResetCmd(RCC_AHB1Periph_GPIOA, ENABLE);
00128     RCC_AHB1PeriphResetCmd(RCC_AHB1Periph_GPIOA, DISABLE);
00129   \}
00130   \textcolor{keywordflow}{else} \textcolor{keywordflow}{if} (GPIOx == GPIOB)
00131   \{
00132     RCC_AHB1PeriphResetCmd(RCC_AHB1Periph_GPIOB, ENABLE);
00133     RCC_AHB1PeriphResetCmd(RCC_AHB1Periph_GPIOB, DISABLE);
00134   \}
00135   \textcolor{keywordflow}{else} \textcolor{keywordflow}{if} (GPIOx == GPIOC)
00136   \{
00137     RCC_AHB1PeriphResetCmd(RCC_AHB1Periph_GPIOC, ENABLE);
00138     RCC_AHB1PeriphResetCmd(RCC_AHB1Periph_GPIOC, DISABLE);
00139   \}
00140   \textcolor{keywordflow}{else} \textcolor{keywordflow}{if} (GPIOx == GPIOD)
00141   \{
00142     RCC_AHB1PeriphResetCmd(RCC_AHB1Periph_GPIOD, ENABLE);
00143     RCC_AHB1PeriphResetCmd(RCC_AHB1Periph_GPIOD, DISABLE);
00144   \}
00145   \textcolor{keywordflow}{else} \textcolor{keywordflow}{if} (GPIOx == GPIOE)
00146   \{
00147     RCC_AHB1PeriphResetCmd(RCC_AHB1Periph_GPIOE, ENABLE);
00148     RCC_AHB1PeriphResetCmd(RCC_AHB1Periph_GPIOE, DISABLE);
00149   \}
00150   \textcolor{keywordflow}{else} \textcolor{keywordflow}{if} (GPIOx == GPIOF)
00151   \{
00152     RCC_AHB1PeriphResetCmd(RCC_AHB1Periph_GPIOF, ENABLE);
00153     RCC_AHB1PeriphResetCmd(RCC_AHB1Periph_GPIOF, DISABLE);
00154   \}
00155   \textcolor{keywordflow}{else} \textcolor{keywordflow}{if} (GPIOx == GPIOG)
00156   \{
00157     RCC_AHB1PeriphResetCmd(RCC_AHB1Periph_GPIOG, ENABLE);
00158     RCC_AHB1PeriphResetCmd(RCC_AHB1Periph_GPIOG, DISABLE);
00159   \}
00160   \textcolor{keywordflow}{else} \textcolor{keywordflow}{if} (GPIOx == GPIOH)
00161   \{
00162     RCC_AHB1PeriphResetCmd(RCC_AHB1Periph_GPIOH, ENABLE);
00163     RCC_AHB1PeriphResetCmd(RCC_AHB1Periph_GPIOH, DISABLE);
00164   \}
00165   \textcolor{keywordflow}{else}
00166   \{
00167     \textcolor{keywordflow}{if} (GPIOx == GPIOI)
00168     \{
00169       RCC_AHB1PeriphResetCmd(RCC_AHB1Periph_GPIOI, ENABLE);
00170       RCC_AHB1PeriphResetCmd(RCC_AHB1Periph_GPIOI, DISABLE);
00171     \}
00172   \}
00173 \}
00174 
00175 \textcolor{comment}{/**}
00176 \textcolor{comment}{  * @brief  Initializes the GPIOx peripheral according to the specified parameters in the
       GPIO\_InitStruct.}
00177 \textcolor{comment}{  * @param  GPIOx: where x can be (A..I) to select the GPIO peripheral.}
00178 \textcolor{comment}{  * @param  GPIO\_InitStruct: pointer to a GPIO\_InitTypeDef structure that contains}
00179 \textcolor{comment}{  *         the configuration information for the specified GPIO peripheral.}
00180 \textcolor{comment}{  * @retval None}
00181 \textcolor{comment}{  */}
00182 \textcolor{keywordtype}{void} GPIO_Init(GPIO\_TypeDef* GPIOx, GPIO\_InitTypeDef* GPIO\_InitStruct)
00183 \{
00184   uint32\_t pinpos = 0x00, pos = 0x00 , currentpin = 0x00;
00185 
00186   \textcolor{comment}{/* Check the parameters */}
00187   assert_param(IS\_GPIO\_ALL\_PERIPH(GPIOx));
00188   assert_param(IS\_GPIO\_PIN(GPIO\_InitStruct->GPIO\_Pin));
00189   assert_param(IS\_GPIO\_MODE(GPIO\_InitStruct->GPIO\_Mode));
00190   assert_param(IS\_GPIO\_PUPD(GPIO\_InitStruct->GPIO\_PuPd));
00191 
00192   \textcolor{comment}{/* -------------------------Configure the port pins---------------- */}
00193   \textcolor{comment}{/*-- GPIO Mode Configuration --*/}
00194   \textcolor{keywordflow}{for} (pinpos = 0x00; pinpos < 0x10; pinpos++)
00195   \{
00196     pos = ((uint32\_t)0x01) << pinpos;
00197     \textcolor{comment}{/* Get the port pins position */}
00198     currentpin = (GPIO\_InitStruct->GPIO_Pin) & pos;
00199 
00200     \textcolor{keywordflow}{if} (currentpin == pos)
00201     \{
00202       GPIOx->MODER  &= ~(GPIO_MODER_MODER0 << (pinpos * 2));
00203       GPIOx->MODER |= (((uint32\_t)GPIO\_InitStruct->GPIO\_Mode) << (pinpos * 2));
00204 
00205       \textcolor{keywordflow}{if} ((GPIO\_InitStruct->GPIO_Mode == GPIO_Mode_OUT) || (GPIO\_InitStruct
      ->GPIO_Mode == GPIO_Mode_AF))
00206       \{
00207         \textcolor{comment}{/* Check Speed mode parameters */}
00208         assert_param(IS\_GPIO\_SPEED(GPIO\_InitStruct->GPIO\_Speed));
00209 
00210         \textcolor{comment}{/* Speed mode configuration */}
00211         GPIOx->OSPEEDR &= ~(GPIO_OSPEEDER_OSPEEDR0 << (pinpos * 2));
00212         GPIOx->OSPEEDR |= ((uint32\_t)(GPIO\_InitStruct->GPIO\_Speed) << (pinpos * 2));
00213 
00214         \textcolor{comment}{/* Check Output mode parameters */}
00215         assert_param(IS\_GPIO\_OTYPE(GPIO\_InitStruct->GPIO\_OType));
00216 
00217         \textcolor{comment}{/* Output mode configuration*/}
00218         GPIOx->OTYPER  &= ~((GPIO_OTYPER_OT_0) << ((uint16\_t)pinpos)) ;
00219         GPIOx->OTYPER |= (uint16\_t)(((uint16\_t)GPIO\_InitStruct->GPIO\_OType) << ((uint16\_t)pinpos));
00220       \}
00221 
00222       \textcolor{comment}{/* Pull-up Pull down resistor configuration*/}
00223       GPIOx->PUPDR &= ~(GPIO_PUPDR_PUPDR0 << ((uint16\_t)pinpos * 2));
00224       GPIOx->PUPDR |= (((uint32\_t)GPIO\_InitStruct->GPIO\_PuPd) << (pinpos * 2));
00225     \}
00226   \}
00227 \}
00228 
00229 \textcolor{comment}{/**}
00230 \textcolor{comment}{  * @brief  Fills each GPIO\_InitStruct member with its default value.}
00231 \textcolor{comment}{  * @param  GPIO\_InitStruct : pointer to a GPIO\_InitTypeDef structure which will be initialized.}
00232 \textcolor{comment}{  * @retval None}
00233 \textcolor{comment}{  */}
00234 \textcolor{keywordtype}{void} GPIO_StructInit(GPIO\_InitTypeDef* GPIO\_InitStruct)
00235 \{
00236   \textcolor{comment}{/* Reset GPIO init structure parameters values */}
00237   GPIO\_InitStruct->GPIO_Pin  = GPIO_Pin_All;
00238   GPIO\_InitStruct->GPIO_Mode = GPIO_Mode_IN;
00239   GPIO\_InitStruct->GPIO_Speed = GPIO_Speed_2MHz;
00240   GPIO\_InitStruct->GPIO_OType = GPIO_OType_PP;
00241   GPIO\_InitStruct->GPIO_PuPd = GPIO_PuPd_NOPULL;
00242 \}
00243 
00244 \textcolor{comment}{/**}
00245 \textcolor{comment}{  * @brief  Locks GPIO Pins configuration registers.}
00246 \textcolor{comment}{  * @note   The locked registers are GPIOx\_MODER, GPIOx\_OTYPER, GPIOx\_OSPEEDR,}
00247 \textcolor{comment}{  *         GPIOx\_PUPDR, GPIOx\_AFRL and GPIOx\_AFRH.}
00248 \textcolor{comment}{  * @note   The configuration of the locked GPIO pins can no longer be modified}
00249 \textcolor{comment}{  *         until the next reset.}
00250 \textcolor{comment}{  * @param  GPIOx: where x can be (A..I) to select the GPIO peripheral.}
00251 \textcolor{comment}{  * @param  GPIO\_Pin: specifies the port bit to be locked.}
00252 \textcolor{comment}{  *          This parameter can be any combination of GPIO\_Pin\_x where x can be (0..15).}
00253 \textcolor{comment}{  * @retval None}
00254 \textcolor{comment}{  */}
00255 \textcolor{keywordtype}{void} GPIO_PinLockConfig(GPIO\_TypeDef* GPIOx, uint16\_t GPIO\_Pin)
00256 \{
00257   \_\_IO uint32\_t tmp = 0x00010000;
00258 
00259   \textcolor{comment}{/* Check the parameters */}
00260   assert_param(IS\_GPIO\_ALL\_PERIPH(GPIOx));
00261   assert_param(IS\_GPIO\_PIN(GPIO\_Pin));
00262 
00263   tmp |= GPIO\_Pin;
00264   \textcolor{comment}{/* Set LCKK bit */}
00265   GPIOx->LCKR = tmp;
00266   \textcolor{comment}{/* Reset LCKK bit */}
00267   GPIOx->LCKR =  GPIO\_Pin;
00268   \textcolor{comment}{/* Set LCKK bit */}
00269   GPIOx->LCKR = tmp;
00270   \textcolor{comment}{/* Read LCKK bit*/}
00271   tmp = GPIOx->LCKR;
00272   \textcolor{comment}{/* Read LCKK bit*/}
00273   tmp = GPIOx->LCKR;
00274 \}
00275 
00276 \textcolor{comment}{/**}
00277 \textcolor{comment}{  * @\}}
00278 \textcolor{comment}{  */}
00279 
00280 \textcolor{comment}{/** @defgroup GPIO\_Group2 GPIO Read and Write}
00281 \textcolor{comment}{ *  @brief   GPIO Read and Write}
00282 \textcolor{comment}{ *}
00283 \textcolor{comment}{@verbatim   }
00284 \textcolor{comment}{ ===============================================================================}
00285 \textcolor{comment}{                              GPIO Read and Write}
00286 \textcolor{comment}{ ===============================================================================  }
00287 \textcolor{comment}{}
00288 \textcolor{comment}{@endverbatim}
00289 \textcolor{comment}{  * @\{}
00290 \textcolor{comment}{  */}
00291 
00292 \textcolor{comment}{/**}
00293 \textcolor{comment}{  * @brief  Reads the specified input port pin.}
00294 \textcolor{comment}{  * @param  GPIOx: where x can be (A..I) to select the GPIO peripheral.}
00295 \textcolor{comment}{  * @param  GPIO\_Pin: specifies the port bit to read.}
00296 \textcolor{comment}{  *         This parameter can be GPIO\_Pin\_x where x can be (0..15).}
00297 \textcolor{comment}{  * @retval The input port pin value.}
00298 \textcolor{comment}{  */}
00299 uint8\_t GPIO_ReadInputDataBit(GPIO\_TypeDef* GPIOx, uint16\_t GPIO\_Pin)
00300 \{
00301   uint8\_t bitstatus = 0x00;
00302 
00303   \textcolor{comment}{/* Check the parameters */}
00304   assert_param(IS\_GPIO\_ALL\_PERIPH(GPIOx));
00305   assert_param(IS\_GET\_GPIO\_PIN(GPIO\_Pin));
00306 
00307   \textcolor{keywordflow}{if} ((GPIOx->IDR & GPIO\_Pin) != (uint32\_t)Bit\_RESET)
00308   \{
00309     bitstatus = (uint8\_t)Bit\_SET;
00310   \}
00311   \textcolor{keywordflow}{else}
00312   \{
00313     bitstatus = (uint8\_t)Bit\_RESET;
00314   \}
00315   \textcolor{keywordflow}{return} bitstatus;
00316 \}
00317 
00318 \textcolor{comment}{/**}
00319 \textcolor{comment}{  * @brief  Reads the specified GPIO input data port.}
00320 \textcolor{comment}{  * @param  GPIOx: where x can be (A..I) to select the GPIO peripheral.}
00321 \textcolor{comment}{  * @retval GPIO input data port value.}
00322 \textcolor{comment}{  */}
00323 uint16\_t GPIO_ReadInputData(GPIO\_TypeDef* GPIOx)
00324 \{
00325   \textcolor{comment}{/* Check the parameters */}
00326   assert_param(IS\_GPIO\_ALL\_PERIPH(GPIOx));
00327 
00328   \textcolor{keywordflow}{return} ((uint16\_t)GPIOx->IDR);
00329 \}
00330 
00331 \textcolor{comment}{/**}
00332 \textcolor{comment}{  * @brief  Reads the specified output data port bit.}
00333 \textcolor{comment}{  * @param  GPIOx: where x can be (A..I) to select the GPIO peripheral.}
00334 \textcolor{comment}{  * @param  GPIO\_Pin: specifies the port bit to read.}
00335 \textcolor{comment}{  *          This parameter can be GPIO\_Pin\_x where x can be (0..15).}
00336 \textcolor{comment}{  * @retval The output port pin value.}
00337 \textcolor{comment}{  */}
00338 uint8\_t GPIO_ReadOutputDataBit(GPIO\_TypeDef* GPIOx, uint16\_t GPIO\_Pin)
00339 \{
00340   uint8\_t bitstatus = 0x00;
00341 
00342   \textcolor{comment}{/* Check the parameters */}
00343   assert_param(IS\_GPIO\_ALL\_PERIPH(GPIOx));
00344   assert_param(IS\_GET\_GPIO\_PIN(GPIO\_Pin));
00345 
00346   \textcolor{keywordflow}{if} ((GPIOx->ODR & GPIO\_Pin) != (uint32\_t)Bit\_RESET)
00347   \{
00348     bitstatus = (uint8\_t)Bit\_SET;
00349   \}
00350   \textcolor{keywordflow}{else}
00351   \{
00352     bitstatus = (uint8\_t)Bit\_RESET;
00353   \}
00354   \textcolor{keywordflow}{return} bitstatus;
00355 \}
00356 
00357 \textcolor{comment}{/**}
00358 \textcolor{comment}{  * @brief  Reads the specified GPIO output data port.}
00359 \textcolor{comment}{  * @param  GPIOx: where x can be (A..I) to select the GPIO peripheral.}
00360 \textcolor{comment}{  * @retval GPIO output data port value.}
00361 \textcolor{comment}{  */}
00362 uint16\_t GPIO_ReadOutputData(GPIO\_TypeDef* GPIOx)
00363 \{
00364   \textcolor{comment}{/* Check the parameters */}
00365   assert_param(IS\_GPIO\_ALL\_PERIPH(GPIOx));
00366 
00367   \textcolor{keywordflow}{return} ((uint16\_t)GPIOx->ODR);
00368 \}
00369 
00370 \textcolor{comment}{/**}
00371 \textcolor{comment}{  * @brief  Sets the selected data port bits.}
00372 \textcolor{comment}{  * @note   This functions uses GPIOx\_BSRR register to allow atomic read/modify }
00373 \textcolor{comment}{  *         accesses. In this way, there is no risk of an IRQ occurring between}
00374 \textcolor{comment}{  *         the read and the modify access.}
00375 \textcolor{comment}{  * @param  GPIOx: where x can be (A..I) to select the GPIO peripheral.}
00376 \textcolor{comment}{  * @param  GPIO\_Pin: specifies the port bits to be written.}
00377 \textcolor{comment}{  *          This parameter can be any combination of GPIO\_Pin\_x where x can be (0..15).}
00378 \textcolor{comment}{  * @retval None}
00379 \textcolor{comment}{  */}
00380 \textcolor{keywordtype}{void} GPIO_SetBits(GPIO\_TypeDef* GPIOx, uint16\_t GPIO\_Pin)
00381 \{
00382   \textcolor{comment}{/* Check the parameters */}
00383   assert_param(IS\_GPIO\_ALL\_PERIPH(GPIOx));
00384   assert_param(IS\_GPIO\_PIN(GPIO\_Pin));
00385 
00386   GPIOx->BSRRL = GPIO\_Pin;
00387 \}
00388 
00389 \textcolor{comment}{/**}
00390 \textcolor{comment}{  * @brief  Clears the selected data port bits.}
00391 \textcolor{comment}{  * @note   This functions uses GPIOx\_BSRR register to allow atomic read/modify }
00392 \textcolor{comment}{  *         accesses. In this way, there is no risk of an IRQ occurring between}
00393 \textcolor{comment}{  *         the read and the modify access.}
00394 \textcolor{comment}{  * @param  GPIOx: where x can be (A..I) to select the GPIO peripheral.}
00395 \textcolor{comment}{  * @param  GPIO\_Pin: specifies the port bits to be written.}
00396 \textcolor{comment}{  *          This parameter can be any combination of GPIO\_Pin\_x where x can be (0..15).}
00397 \textcolor{comment}{  * @retval None}
00398 \textcolor{comment}{  */}
00399 \textcolor{keywordtype}{void} GPIO_ResetBits(GPIO\_TypeDef* GPIOx, uint16\_t GPIO\_Pin)
00400 \{
00401   \textcolor{comment}{/* Check the parameters */}
00402   assert_param(IS\_GPIO\_ALL\_PERIPH(GPIOx));
00403   assert_param(IS\_GPIO\_PIN(GPIO\_Pin));
00404 
00405   GPIOx->BSRRH = GPIO\_Pin;
00406 \}
00407 
00408 \textcolor{comment}{/**}
00409 \textcolor{comment}{  * @brief  Sets or clears the selected data port bit.}
00410 \textcolor{comment}{  * @param  GPIOx: where x can be (A..I) to select the GPIO peripheral.}
00411 \textcolor{comment}{  * @param  GPIO\_Pin: specifies the port bit to be written.}
00412 \textcolor{comment}{  *          This parameter can be one of GPIO\_Pin\_x where x can be (0..15).}
00413 \textcolor{comment}{  * @param  BitVal: specifies the value to be written to the selected bit.}
00414 \textcolor{comment}{  *          This parameter can be one of the BitAction enum values:}
00415 \textcolor{comment}{  *            @arg Bit\_RESET: to clear the port pin}
00416 \textcolor{comment}{  *            @arg Bit\_SET: to set the port pin}
00417 \textcolor{comment}{  * @retval None}
00418 \textcolor{comment}{  */}
00419 \textcolor{keywordtype}{void} GPIO_WriteBit(GPIO\_TypeDef* GPIOx, uint16\_t GPIO\_Pin, BitAction BitVal)
00420 \{
00421   \textcolor{comment}{/* Check the parameters */}
00422   assert_param(IS\_GPIO\_ALL\_PERIPH(GPIOx));
00423   assert_param(IS\_GET\_GPIO\_PIN(GPIO\_Pin));
00424   assert_param(IS\_GPIO\_BIT\_ACTION(BitVal));
00425 
00426   \textcolor{keywordflow}{if} (BitVal != Bit_RESET)
00427   \{
00428     GPIOx->BSRRL = GPIO\_Pin;
00429   \}
00430   \textcolor{keywordflow}{else}
00431   \{
00432     GPIOx->BSRRH = GPIO\_Pin ;
00433   \}
00434 \}
00435 
00436 \textcolor{comment}{/**}
00437 \textcolor{comment}{  * @brief  Writes data to the specified GPIO data port.}
00438 \textcolor{comment}{  * @param  GPIOx: where x can be (A..I) to select the GPIO peripheral.}
00439 \textcolor{comment}{  * @param  PortVal: specifies the value to be written to the port output data register.}
00440 \textcolor{comment}{  * @retval None}
00441 \textcolor{comment}{  */}
00442 \textcolor{keywordtype}{void} GPIO_Write(GPIO\_TypeDef* GPIOx, uint16\_t PortVal)
00443 \{
00444   \textcolor{comment}{/* Check the parameters */}
00445   assert_param(IS\_GPIO\_ALL\_PERIPH(GPIOx));
00446 
00447   GPIOx->ODR = PortVal;
00448 \}
00449 
00450 \textcolor{comment}{/**}
00451 \textcolor{comment}{  * @brief  Toggles the specified GPIO pins..}
00452 \textcolor{comment}{  * @param  GPIOx: where x can be (A..I) to select the GPIO peripheral.}
00453 \textcolor{comment}{  * @param  GPIO\_Pin: Specifies the pins to be toggled.}
00454 \textcolor{comment}{  * @retval None}
00455 \textcolor{comment}{  */}
00456 \textcolor{keywordtype}{void} GPIO_ToggleBits(GPIO\_TypeDef* GPIOx, uint16\_t GPIO\_Pin)
00457 \{
00458   \textcolor{comment}{/* Check the parameters */}
00459   assert_param(IS\_GPIO\_ALL\_PERIPH(GPIOx));
00460 
00461   GPIOx->ODR ^= GPIO\_Pin;
00462 \}
00463 
00464 \textcolor{comment}{/**}
00465 \textcolor{comment}{  * @\}}
00466 \textcolor{comment}{  */}
00467 
00468 \textcolor{comment}{/** @defgroup GPIO\_Group3 GPIO Alternate functions configuration function}
00469 \textcolor{comment}{ *  @brief   GPIO Alternate functions configuration function}
00470 \textcolor{comment}{ *}
00471 \textcolor{comment}{@verbatim   }
00472 \textcolor{comment}{ ===============================================================================}
00473 \textcolor{comment}{               GPIO Alternate functions configuration function}
00474 \textcolor{comment}{ ===============================================================================  }
00475 \textcolor{comment}{}
00476 \textcolor{comment}{@endverbatim}
00477 \textcolor{comment}{  * @\{}
00478 \textcolor{comment}{  */}
00479 
00480 \textcolor{comment}{/**}
00481 \textcolor{comment}{  * @brief  Changes the mapping of the specified pin.}
00482 \textcolor{comment}{  * @param  GPIOx: where x can be (A..I) to select the GPIO peripheral.}
00483 \textcolor{comment}{  * @param  GPIO\_PinSource: specifies the pin for the Alternate function.}
00484 \textcolor{comment}{  *         This parameter can be GPIO\_PinSourcex where x can be (0..15).}
00485 \textcolor{comment}{  * @param  GPIO\_AFSelection: selects the pin to used as Alternate function.}
00486 \textcolor{comment}{  *          This parameter can be one of the following values:}
00487 \textcolor{comment}{  *            @arg GPIO\_AF\_RTC\_50Hz: Connect RTC\_50Hz pin to AF0 (default after reset) }
00488 \textcolor{comment}{  *            @arg GPIO\_AF\_MCO: Connect MCO pin (MCO1 and MCO2) to AF0 (default after reset) }
00489 \textcolor{comment}{  *            @arg GPIO\_AF\_TAMPER: Connect TAMPER pins (TAMPER\_1 and TAMPER\_2) to AF0 (default after
       reset) }
00490 \textcolor{comment}{  *            @arg GPIO\_AF\_SWJ: Connect SWJ pins (SWD and JTAG)to AF0 (default after reset) }
00491 \textcolor{comment}{  *            @arg GPIO\_AF\_TRACE: Connect TRACE pins to AF0 (default after reset)}
00492 \textcolor{comment}{  *            @arg GPIO\_AF\_TIM1: Connect TIM1 pins to AF1}
00493 \textcolor{comment}{  *            @arg GPIO\_AF\_TIM2: Connect TIM2 pins to AF1}
00494 \textcolor{comment}{  *            @arg GPIO\_AF\_TIM3: Connect TIM3 pins to AF2}
00495 \textcolor{comment}{  *            @arg GPIO\_AF\_TIM4: Connect TIM4 pins to AF2}
00496 \textcolor{comment}{  *            @arg GPIO\_AF\_TIM5: Connect TIM5 pins to AF2}
00497 \textcolor{comment}{  *            @arg GPIO\_AF\_TIM8: Connect TIM8 pins to AF3}
00498 \textcolor{comment}{  *            @arg GPIO\_AF\_TIM9: Connect TIM9 pins to AF3}
00499 \textcolor{comment}{  *            @arg GPIO\_AF\_TIM10: Connect TIM10 pins to AF3}
00500 \textcolor{comment}{  *            @arg GPIO\_AF\_TIM11: Connect TIM11 pins to AF3}
00501 \textcolor{comment}{  *            @arg GPIO\_AF\_I2C1: Connect I2C1 pins to AF4}
00502 \textcolor{comment}{  *            @arg GPIO\_AF\_I2C2: Connect I2C2 pins to AF4}
00503 \textcolor{comment}{  *            @arg GPIO\_AF\_I2C3: Connect I2C3 pins to AF4}
00504 \textcolor{comment}{  *            @arg GPIO\_AF\_SPI1: Connect SPI1 pins to AF5}
00505 \textcolor{comment}{  *            @arg GPIO\_AF\_SPI2: Connect SPI2/I2S2 pins to AF5}
00506 \textcolor{comment}{  *            @arg GPIO\_AF\_SPI3: Connect SPI3/I2S3 pins to AF6}
00507 \textcolor{comment}{  *            @arg GPIO\_AF\_I2S3ext: Connect I2S3ext pins to AF7}
00508 \textcolor{comment}{  *            @arg GPIO\_AF\_USART1: Connect USART1 pins to AF7}
00509 \textcolor{comment}{  *            @arg GPIO\_AF\_USART2: Connect USART2 pins to AF7}
00510 \textcolor{comment}{  *            @arg GPIO\_AF\_USART3: Connect USART3 pins to AF7}
00511 \textcolor{comment}{  *            @arg GPIO\_AF\_UART4: Connect UART4 pins to AF8}
00512 \textcolor{comment}{  *            @arg GPIO\_AF\_UART5: Connect UART5 pins to AF8}
00513 \textcolor{comment}{  *            @arg GPIO\_AF\_USART6: Connect USART6 pins to AF8}
00514 \textcolor{comment}{  *            @arg GPIO\_AF\_CAN1: Connect CAN1 pins to AF9}
00515 \textcolor{comment}{  *            @arg GPIO\_AF\_CAN2: Connect CAN2 pins to AF9}
00516 \textcolor{comment}{  *            @arg GPIO\_AF\_TIM12: Connect TIM12 pins to AF9}
00517 \textcolor{comment}{  *            @arg GPIO\_AF\_TIM13: Connect TIM13 pins to AF9}
00518 \textcolor{comment}{  *            @arg GPIO\_AF\_TIM14: Connect TIM14 pins to AF9}
00519 \textcolor{comment}{  *            @arg GPIO\_AF\_OTG\_FS: Connect OTG\_FS pins to AF10}
00520 \textcolor{comment}{  *            @arg GPIO\_AF\_OTG\_HS: Connect OTG\_HS pins to AF10}
00521 \textcolor{comment}{  *            @arg GPIO\_AF\_ETH: Connect ETHERNET pins to AF11}
00522 \textcolor{comment}{  *            @arg GPIO\_AF\_FSMC: Connect FSMC pins to AF12}
00523 \textcolor{comment}{  *            @arg GPIO\_AF\_OTG\_HS\_FS: Connect OTG HS (configured in FS) pins to AF12}
00524 \textcolor{comment}{  *            @arg GPIO\_AF\_SDIO: Connect SDIO pins to AF12}
00525 \textcolor{comment}{  *            @arg GPIO\_AF\_DCMI: Connect DCMI pins to AF13}
00526 \textcolor{comment}{  *            @arg GPIO\_AF\_EVENTOUT: Connect EVENTOUT pins to AF15}
00527 \textcolor{comment}{  * @retval None}
00528 \textcolor{comment}{  */}
00529 \textcolor{keywordtype}{void} GPIO_PinAFConfig(GPIO\_TypeDef* GPIOx, uint16\_t GPIO\_PinSource, uint8\_t GPIO\_AF)
00530 \{
00531   uint32\_t temp = 0x00;
00532   uint32\_t temp\_2 = 0x00;
00533 
00534   \textcolor{comment}{/* Check the parameters */}
00535   assert_param(IS\_GPIO\_ALL\_PERIPH(GPIOx));
00536   assert_param(IS\_GPIO\_PIN\_SOURCE(GPIO\_PinSource));
00537   assert_param(IS\_GPIO\_AF(GPIO\_AF));
00538 
00539   temp = ((uint32\_t)(GPIO\_AF) << ((uint32\_t)((uint32\_t)GPIO\_PinSource & (uint32\_t)0x07) * 4)) ;
00540   GPIOx->AFR[GPIO\_PinSource >> 0x03] &= ~((uint32\_t)0xF << ((uint32\_t)((uint32\_t)GPIO\_PinSource & (
      uint32\_t)0x07) * 4)) ;
00541   temp\_2 = GPIOx->AFR[GPIO\_PinSource >> 0x03] | temp;
00542   GPIOx->AFR[GPIO\_PinSource >> 0x03] = temp\_2;
00543 \}
00544 
00545 \textcolor{comment}{/**}
00546 \textcolor{comment}{  * @\}}
00547 \textcolor{comment}{  */}
00548 
00549 \textcolor{comment}{/**}
00550 \textcolor{comment}{  * @\}}
00551 \textcolor{comment}{  */}
00552 
00553 \textcolor{comment}{/**}
00554 \textcolor{comment}{  * @\}}
00555 \textcolor{comment}{  */}
00556 
00557 \textcolor{comment}{/**}
00558 \textcolor{comment}{  * @\}}
00559 \textcolor{comment}{  */}
00560 
00561 \textcolor{comment}{/******************* (C) COPYRIGHT 2011 STMicroelectronics *****END OF FILE****/}
\end{DoxyCode}
